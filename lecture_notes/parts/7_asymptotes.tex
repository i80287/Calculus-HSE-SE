\chapter{Асимптоты}

\section{Определения асимптот}

\mcdfn{Асимптоты}
{
\begin{tabular}{rl}
    Вертикальная асимптота:   & $\bullet$ Прямая $ x = a $ называется вертикальной асимптотой \\
                              & для графика функции $y=f(x)$, если \\
                              & $ \lim_{x \to a-} f(x) = \pm \infty \vee \lim_{x \to a+} f(x) = \pm \infty $ \\
    Горизонтальная асимптота: & $\bullet$ Прямая $y=b$ называется горизонтальной асимптотой для \\
                              & графика функции $y=f(x)$ на $\pm\infty$, если \\
                              & $ \lim_{x \to \pm\infty} f(x) = b $ \\
                              & Вообще говоря, горизонтальные асимптоты на $+\infty$ и $-\infty$ \\
                              & могут быть разными \\
    Наклонная асимптота:      & $\bullet$ Прямая $y=kx + b$ называется наклонной асимптотой для \\
                              & графика функции $y=f(x)$ при $x \to \pm\infty$, если \\
                              & $ \lim_{x \to \pm\infty} f(x) - (kx + b) = 0 $ \\
                              & Вообще говоря, наклонные асимптоты на $+\infty$ и $-\infty$ \\
                              & могут быть разными \\
\end{tabular}
}

\section{Признак наклонной асимптоты}

\mcthm{Признак наклонной асимптоты}
{
    Прямая $y=kx + b$ - наклонная асимптота графика функции $y=f(x)$ при $x \to +\infty \iff  $
$$
\left\{ \begin{tabular}{l}
    $ \lim_{x \to +\infty} \frac{f(x)}{x} = k $ \\
    $ \lim_{x \to +\infty} f(x) - kx = b $
\end{tabular} \right.
$$

\mcprf{
\begin{split}
    & "\implies" \\
    & \text{1. Распишем определение наклонной асимптоты: } \lim_{x \to +\infty} (f(x) - (kx + b)) = 0 \\
    & \text{Вынесем $b$ из предела: } \lim_{x \to +\infty} f(x) - kx = b \\
    & f(x) - kx - b \text{ - б.м. при } x \to +\infty \\
    & \text{Т.к. } x \to +\infty \text{, то можно поделить на } x: \\
    & \frac{f(x)}{x} - k = \frac{b}{x} + \frac{\text{б.м.}}{x} \\
    & \left. \begin{tabular}{l}
        $ \frac{b}{x} \underset{x \to+\infty}{\to} 0 $ \\
        $ \frac{\text{б.м.}}{x} \underset{x \to+\infty}{\to} 0 $
    \end{tabular} \right\}
    \begin{tabular}{l}
        $ \implies \frac{f(x)}{x} - k \underset{x \to+\infty}{\to} 0 $
    \end{tabular} \\
    & "\impliedby" \\
    & \text{Т.к. } \lim_{x\to +\infty} f(x) - kx = b \text{, то } \lim_{x\to +\infty} (f(x) - (kx + b)) = 0 \\
\end{split}
}
}
