\chapter{Определения и свойства числовых последовательностей}

\section{Определения}

\subsection{Числовая последовательность}

\mcdfn{Числовая последовательность}
{
    Числовая последовательность - это счётно бесконечный проиндексированный набор чисел
}

\subsection{Ограниченная ч.п.}

\mcdfn{Ограниченная сверху числовая последовательность}
{
    Числовая последовательность $ \NumSeq{a_n} $ называется ограниченной сверху, если

    $ \exists C \in \mathbb{R} \, \forall n \in \mathbb{N}: \, a_n < C $
}

\mcdfn{Ограниченная снизу числовая последовательность}
{
    Числовая последовательность $ \NumSeq{a_n} $ называется ограниченной снизу, если

    $ \exists C \in \mathbb{R} \, \forall n \in \mathbb{N}: \, a_n > -C $
}

\mcdfn{Ограниченная числовая последовательность}
{
    Числовая последовательность $ \NumSeq{a_n} $ называется ограниченной, если

    $ \exists C > 0 \, \forall n \in \mathbb{N}: \, | a_n | < C $
}

\nt
{
    Числовая последовательность ограничена $ \iff $ она ограничена сверху и ограничена снизу
}

\subsection{Неограниченная ч.п.}

\mcdfn{Неограниченная числовая последовательность}
{
    Числовая последовательность $ \NumSeq{a_n} $ называется неограниченной, если она не является ограниченной, то есть

    $ \forall C > 0 \, \exists n \in \mathbb{N}: \, | a_n | \ge C $
}

\subsection{Отделимая от нуля ч.п.}

\mcdfn{Отделимая от нуля числовая последовательность}
{
    Числовая последовательность $ \NumSeq{a_n} $ называется отделимой от нуля, если

    $ \exists \veps > 0 \, \forall n \in \mathbb{N}: \, | a_n | > \veps $
}

\subsection{Эпсилон окрестность}

\mcdfn{Эпсилон окрестность}
{
    Эпсилон окрестностью вещественного числа $ x_0 $ (элемента поля вещественных чисел)
    называется множество $ (x_0 - \veps; x_0 + \veps) $ и обозначается $ U_{\veps}(x_0) $.

    Обычно говорят "Эпсилон окрестность точки $ x_0 $"
}

\mcex{}{
    $ U_1(\pi) = (\pi - 1; \pi + 1) $

    $ U_e(e) = (0; 2 e) $
}

\mcdfn{Проколотая эпсилон окрестность}
{
    Проколотой эпсилон окрестностью вещественного числа $ x_0 $ (элемента поля вещественных чисел)
    называется множество $ (x_0 - \veps; x_0 + \veps) \setminus \{ x_0 \} $ и обозначается $ \dot{U}_{\veps}(x_0) $.

    Обычно говорят "Проколотая эпсилон окрестность точки $ x_0 $"
}

\mcex{}{
    $ \dot{U}_1(e) = (e - 1; e + 1) \setminus \{ e \} = (e - 1; e) \cup (e; e + 1) $
}

\subsection{Сходящаяся ч.п.}

\mcdfn{Сходящаяся числовая последовательность}
{
    Числовая последовательность называется сходящейся, если она имеет конечный предел при $ n \to +\infty $, т.е.
    ч.п. $ \NumSeq{a_n} $ называется сходящейся, если $ \exists \lim_{n \to +\infty} a_n = A \in \mathbb{R} $, то есть по определению
    \[ \exists A \in \mathbb{R} \, \forall \veps > 0 \, \exists N = N(\veps) \forall n > N: | a_n - A | < \veps \]
}

\nt{
    Сходящаяся ч.п. является ограниченной
}

\nt{
    Неравенство $ | a_n - A | < \veps $ равносильно тому, что $ a_n \in U_\veps(A) $
}

\subsection{Бесконечно большая ч.п.}

\mcdfn{Бесконечно большая числовая последовательность}
{
    Числовая последовательность $ \NumSeq{a_n} $ называется бесконечно большой, если она стремится к $ +\infty$, к $ -\infty $ или к $ \infty $ при $ n \to +\infty $, т.е.

\begin{tabular}{rl}
    & $\bullet$
        $ \lim_{n \to +\infty} a_n = +\infty \iff
        \forall M > 0 \, \exists N = N(M) \forall n > N: a_n > M $ \\
    & $\bullet$
        $ \lim_{n \to +\infty} a_n = -\infty \iff
        \forall M > 0 \, \exists N = N(M) \forall n > N: a_n < -M $ \\
    & $\bullet$
        $ \lim_{n \to +\infty} a_n = \infty \iff
        \forall M > 0 \, \exists N = N(M) \forall n > N: | a_n | > M $ \\
\end{tabular}
}

\subsection{Бесконечно малая ч.п.}

\mcdfn{Бесконечно малая числовая последовательность}
{
    Числовая последовательность $ \NumSeq{a_n} $ называется бесконечно малой, если она стремится к 0 при $ n \to +\infty $, т.е.

    $ \forall \veps > 0 \exists N = N(\veps) \forall n > N: | a_n | < \veps $
}

\subsection{Связи числовых последовательностей}

\nt{
\begin{tabular}{rl}
    Связи числовых последовательностей:

    & $\bullet$ $ \frac{1}{\text{б.б.}} = \text{б.м.} $ \\
    & $\bullet$ $ \frac{1}{\text{б.м.}} = \text{б.б.} $ \\
    & $\bullet$ $ \frac{1}{\text{ограниченная}} = \text{отделимая от нуля} $ \\
    & $\bullet$ $ \frac{1}{\text{отделимая от нуля}} = \text{ограниченная} $ \\
\end{tabular}
}

\nt
{
    Если ч.п. сходится или является б.б., то предел единственный
}

\mcprop{Докажите по определению, что}
{
    (ограниченная ч.п.) + (ограниченная ч.п.) = ограниченная ч.п.

    б.м + б.м. = б.м.

    б.м. $\cdot$ (ограниченная ч.п.) = б.м.

    $\frac{\text{отделимая от нуля ч.п.}}{\text{ограниченная ч.п.}}$ = ограничена ч.п.
}

\mcprop{Приведите пример, когда}
{
    (отделимая от нуля ч.п.) + (отделимая от нуля ч.п.) = отделимая от нуля ч.п.

    (отделимая от нуля ч.п.) + (отделимая от нуля ч.п.) = б.м.

    б.б + б.б = б.б.

    б.б + б.б = б.м.

    б.б + б.б = (ограниченная ч.п.)

    б.б + б.б = (отделимая от нуля ч.п.)
}

\section{Теоремы}

\subsection{Теорема о предельном переходе в неравенствах}

\mcthm{Теорема: свойство предельного перехода в неравенствах}
{
    \[ (\exists N \in \mathbb{N} \, \forall n \ge N: c_n > A) \wedge (\lim_{n\to\infty} c_n = C) \implies C \ge A \]
\mcprf{
\begin{split}
    & \text{1. Распишем, что дано, по определению:} \\
    & \forall \veps > 0 \exists N_1(\veps) \forall n > N_1(\veps): | c_n - C | < \veps \\
    & \text{Это равносильно } \forall \veps > 0 \exists N_1(\veps) \forall n > N_1(\veps): C - \veps < c_n < C + \veps \\
    & \exists N \in \mathbb{N} \, \forall n \ge N: c_n > A \\
    & \text{2. Для любого $ \veps $ рассмотрим } M(\veps) = \max(N_1(\veps), N) + 1 \\
    & \text{Тогда } \forall \veps > 0 \exists M(\veps) = \max(N_1(\veps), N) + 1 \, \forall n > M: (C - \veps < c_n < C + \veps \wedge c_n > A) \\
    & \text{Следовательно, } \forall \veps > 0 \exists M(\veps) \, \forall n > M: C + \veps > A \\
    & \text{Выражение под кванторами не зависит от $M$ и $n$ } \implies \forall \veps > 0: C + \veps > A \\
    & \text{3. Предположим от противного, что } C < A \\
    & \text{Положим } \veps := \frac{A - C}{2} > 0 \implies C + \veps = C + \frac{A - C}{2} = \frac{A + C}{2} < A \\
    & \text{Получили, что }
        \exists \veps > 0: C + \veps < A
        \implies \circled{$\mathbb{W}$} \implies \text{ предположение, что $ C < A $, неверно}
        \implies C \ge A \\ 
\end{split}
}
}

\subsection{Теорема о зажатой последовательности}

\mcthm{Теорема о зажатой последовательности (о 2 миллиционерах / 2 полицейских / гамбургерах)}
{
\begin{minipage}[t]{\textwidth}
$$
\left. \begin{tabular}{l}
    $ a_n, b_n, c_n $ - числовые последовательности \\
    $ \lim_{n\to\infty} a_n = X $ \\
    $ \lim_{n\to\infty} b_n = X $ \\
    $ \exists N \in \mathbb{N} \hspace{2pt} \forall n \ge N: a_n \le c_n \le b_n $
\end{tabular} \right\}
\begin{tabular}{l}
    $ \implies \lim_{n\to\infty} c_n = X $
\end{tabular}
$$
\end{minipage}
\begin{mcproof}
\begin{equation*}
\begin{split}
    & \text{Докажем для случая, когда } X \in \mathbb{R}
        \text{. При } X \in \overline{\mathbb{R}} \setminus \mathbb{R}
        \text{ доказательство проводится аналогично} \\
    & \text{1. Распишем по определению пределы.} \\
    & \forall \veps > 0 \, \exists N_1(\veps) \, \forall n > N_1(\veps): X - \veps < a_n < X + \veps \\
    & \forall \veps > 0 \, \exists N_2(\veps) \, \forall n > N_2(\veps): X - \veps < b_n < X + \veps \\
    & \text{Рассмотрим } N_3(\veps) = \max(N_1(\veps), N_2(\veps), N) \text{, тогда } \\
    & \forall \veps > 0 \, \exists N_3(\veps) \, \forall n > N_3(\veps): X - \veps < a_n \le c_n \le b_n < X + \veps \\
    & \implies \forall \veps > 0 \, \exists N_3(\veps) \, \forall n > N_3(\veps): X - \veps < c_n < X + \veps \\
\end{split}
\end{equation*}
\end{mcproof}
}

\subsection{Теорема о свойстве предела б.м. ч.п.}

\mcthm{Теорема о свойстве предела б.м. ч.п.}
{
    если $ a \in \mathbb{R} $, то \[\lim_{n\to\infty} a_n = a \iff a_n = a + \alpha_n, \text{где } \alpha_n \text{ - б.м. ч.п.} \]
\mcprf{
\begin{split}
    & "\implies" \\
    & \text{Распишем по определению, что дано:} \\
    & \lim_{n\to\infty} a_n = a \iff \forall \veps > 0 \, \exists N(\veps) \, \forall n > N(\veps): | a_n - a | < \veps \\
    & \text{Обозначим ч.п. } \alpha_n = a_n - a \text{, тогда } a_n = a + \alpha_n \\
    & \text{Тогда: } \forall \veps > 0 \, \exists N(\veps) \, \forall n > N(\veps): | \alpha_n | < \veps \\
    & \text{Доказали, что } a_n = a + \alpha_n \text{, где } \alpha_n \text{ - б.м. ч.п.} \\
    & "\impliedby" \\
    & \text{Распишем то, что $ \alpha_n $ - б.м., по определению:} \\
    & \lim_{n\to\infty} a_n = a \iff \forall \veps > 0 \, \exists N(\veps) \, \forall n > N(\veps): | \alpha_n | < \veps \\
    & \text{По условию } a_n = a + \alpha_n \text{, тогда } a_n - a = \alpha_n \text{, подставим в выражение под кванторами:} \\
    & \forall \veps > 0 \, \exists N(\veps) \, \forall n > N(\veps): | a_n - a | < \veps \\
    & \text{Доказали по определению, что } \lim_{n\to\infty} a_n = a \\
\end{split}
}
}

\section{Определение монотонности числовой последовательности}

\mcdfn{Монотонность ч.п.}
{
    Ч.п. $ \NumSeq{a_n} $ называется строго возрастающей, если $ \forall n \in \mathbb{N}: a_{n + 1} > a_{n} $

    Ч.п. $ \NumSeq{a_n} $ называется строго убывающей, если $ \forall n \in \mathbb{N}: a_{n + 1} < a_{n} $

    Ч.п. $ \{ a_n \} $ называется неубывающей, если $ \forall n \in \mathbb{N}: a_{n + 1} \ge a_{n} $

    Ч.п. $ \{ a_n \} $ называется невозрастающей, если $ \forall n \in \mathbb{N}: a_{n + 1} \le a_{n} $
}
