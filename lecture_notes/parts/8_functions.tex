\chapter{Определение и свойства функции}

\section{Определения}

\mcdfn{Определение функции}{
    Множество пар $\{ (x, y) \in \mathbb{R}^2 | x \in D_f \wedge y \in E_f \} $ называется 
    функцией $f$ с областью определения $D_f$ и областью значения $E_f$, если 
    $ \forall x \in D_f \, \exists! y \in E_f: (x, y) \in f $ (для удобства  $ (x, y) \in f $ обозначают как $ f(x) = y $)

    Обозначение функции: $ f: X \to Y $

    В данном обозначении подразумевают, что $ D_f = X, E_f \subseteq Y $
}

\mcex{}{
    $ f: \mathbb{N} \cup \{ 0 \} \to \mathbb{R} $

    $ \forall n \in \mathbb{N} \cup \{ 0 \}: f(n) = (-1)^{n + 1} \cdot \left\lceil \frac{n}{2} \right\rceil $, 
    в данном случае $ D_f = \mathbb{N} \cup \{ 0 \}, E_f = \mathbb{Z} \subset \mathbb{R} $

    Т.к. несложно установить, что $E_f = \mathbb{Z}$, то можно написать $ f: \mathbb{N} \cup \{ 0 \} \to \mathbb{Z} $
}

\mcdfn{Определение инъективной функции}{
    Функция $f$ называется инъективной, если $ \forall y \in E_f \, \exists! x \in D_f: f(x) = y $

    Это эквивалентно тому, что $ \forall x_1, x_2 \in D_f: (x_1 \ne x_2 \implies f(x_1) \ne f(x_2)) $

    (говорят, что $f$ - инъекция)
}

\mcex{} {
    $ \forall n \in \mathbb{N} $ функция $f(x) = x^{2 n - 1}$ является инъективной

    $ \forall n \in \mathbb{N} $ функция $f(x) = x^{2 n}$ не является инъективной
}

\mcdfn{Определение сюръективной функции}{
    Функция $f: X \to Y$ называется сюръективной для множества $Y$, если $ E_f = Y $

    (говорят, что $f$ - сюръекция)
    
    Когда говорят, что $f$ сюръективна, не уточняя множество, то подразумевают, что $f$ сюръективна для $Y$

}

\mcex{}{
    Функция $\sin: \mathbb{R} \to \mathbb{R} $ не сюръективна для $\mathbb{R}$, но сюръективна для $[-1; 1]$
}

\mcdfn{Определение биективной функции}{
    Функция $f: X \to Y$ называется биективной, если она инъективна и сюръективна

    (говорят, что $f$ - биекция)
}

\mcex{}{
    Функция $ f: \mathbb{N} \cup \{ 0 \} \to \mathbb{Z} $, такая что \\
    $ \forall n \in \mathbb{N} \cup \{ 0 \}: f(n) = (-1)^{n + 1} \cdot \left\lceil \frac{n}{2} \right\rceil $
    - биекция между $\mathbb{N} \cup \{ 0 \}$ и $\mathbb{Z}$

    (как следствие, показали, что $\mathbb{N} \cup \{ 0 \} \sim \mathbb{Z}$, т.е. множества равномощны)
}

\mcdfn{Определение обратной функции}{
    Функция $y = f^{-1}(x)$ называется обратной функцией к функции $ y = f(x) $,
    если множество пар фукнции $ f^{-1} $ является симметрией множества пар $ f $
}

\nt{
    Функция обратима $\iff$ она инъективна
}

\section{Пределы}

\mcdfn{Определение предела функции по Коши}
{
    \[
    \lim_{x \to x_0} f(x) = A \iff
    \forall \veps > 0 \,
    \exists \delta = \delta(\veps) \,
    \forall x \in \dot{U}_\delta(x_0):
    f(x) \in U_\veps(A) \]
}

\nt
{
    При этом
    $ \dot{U}_\delta(+\infty) = (\delta; +\infty) $,
    $ \dot{U}_\delta(-\infty) = (-\infty; \delta) $,
    $ \dot{U}_\delta(\infty) = (-\infty; \delta) \cup (\delta; +\infty) $
}

\mcdfn{Определение предела функции по Гейне}
{
    \[
    \lim_{x \to x_0} f(x) = A \iff
    \forall \NumSeq{x_n}:
    (x_n \ne x_0 \wedge \lim_{n \to +\infty} x_n = x_0 \implies \lim_{n \to +\infty} f(x_n) = A)\]
}

\mcdfn{Односторонний предел функции}
{
    Левосторонним пределом функции называют предел функции по Коши $ f $ при $ x \to x_0 $ слева, то есть

    \[
    \lim_{x \to x_0-} f(x) = A \iff
    \forall \veps > 0
    \exists \delta = \delta(\veps)   
    \forall x \in (x_0 - \delta; x_0):
    f(x) \in U_\veps(A) \]

    Правосторонним пределом функции называют предел функции по Коши $ f $ при $ x \to x_0 $ справа, то есть

    \[
    \lim_{x \to x_0+} f(x) = A \iff
    \forall \veps > 0
    \exists \delta = \delta(\veps)   
    \forall x \in (x_0; x_0 + \delta):
    f(x) \in U_\veps(A) \]
}

\mcthm{Свойство предела функции при $ x \to x_0, x_0 \in \mathbb{R} $}
{
    \[ \lim_{x \to x_0} f(x) = A \iff \lim_{x \to x_0+} f(x) = \lim_{x \to x_0-} f(x) = A \text{, где } A \in \overline{\mathbb{R}} \]

\mcprf{
\begin{split}
    & "\implies" \\
    & \text{Дано:}
        \forall \veps > 0 \,
            \exists \delta = \delta(\veps) > 0 \,
            \forall x \in \dot{U}_\delta(x_0): \,
            f(x) \in U_\veps(A) \\
    & \text{Тогда:} \\
    & \forall \veps > 0 \,
        \exists \delta = \delta(\veps) > 0 \,
        \forall x \in (x_0; x_0 + \delta): \,
        f(x) \in U_\veps(A) \\
    & \forall \veps > 0 \,
        \exists \delta = \delta(\veps) > 0 \,
        \forall x \in (x_0 - \delta; x_0): \,
        f(x) \in U_\veps(A) \\
    & "\impliedby" \\
    & \text{Дано:} \\
    & \forall \veps > 0 \,
        \exists \delta_1 = \delta_1(\veps) > 0 \,
        \forall x \in (x_0; x_0 + \delta_1): \,
        f(x) \in U_\veps(A) \\
    & \forall \veps > 0 \,
        \exists \delta_2 = \delta_2(\veps) > 0 \,
        \forall x \in (x_0 - \delta_2; x_0): \,
        f(x) \in U_\veps(A) \\
    & \text{Положим } \delta(\veps) = \min(\delta_1(\veps), \delta_2(\veps)) \text{, тогда:} \\
    & \forall \veps > 0 \,
        \exists \delta = \delta(\veps) > 0 \,
            \forall x \in \dot{U}_\delta(x_0) \subseteq (x_0 - \delta_2; x_0) \cup (x_0; x_0 + \delta_1): \,
            f(x) \in U_\veps(A) \\
\end{split}
}
}

\mcdfn{Бесконечные пределы}
{
    
\begin{tabular}{rl}
    & $\bullet$
        $ \lim_{x \to x_0} f(x) = +\infty \iff
        \forall M > 0 \, \exists \delta(M) > 0 \, \forall x \in \dot{U}_{\delta}(x_0): f(x) > M $ \\
    & $\bullet$
        $ \lim_{x \to x_0} f(x) = -\infty \iff
        \forall M > 0 \, \exists \delta(M) > 0 \, \forall x \in \dot{U}_{\delta}(x_0): f(x) < -M $ \\
    & $\bullet$
        $ \lim_{x \to x_0} f(x) = \infty \iff
        \forall M > 0 \, \exists \delta(M) > 0 \, \forall x \in \dot{U}_{\delta}(x_0): | f(x) | > M $ \\
\end{tabular}
}

\mcdfn{Бесконечно малая функция}
{
    Функция называется б.м. при $x \to x_0$, если $\lim_{x \to x_0} f(x) = 0$, при этом $ x_0 \in \mathbb{R} $

    Функция называется б.м. при $x \to +\infty$, если $\lim_{x \to +\infty} f(x) = 0$

    Функция называется б.м. при $x \to -\infty$, если $\lim_{x \to -\infty} f(x) = 0$
}

\mcdfn{Бесконечно большая функция}
{
    Функция называется б.б. при $x \to x_0$, если $\lim_{x \to x_0} f(x) = \infty $, при этом $ x_0 \in \mathbb{R} $

    Функция называется б.б. при $x \to +\infty$, если $\lim_{x \to +\infty} f(x) = \infty $

    Функция называется б.б. при $x \to -\infty$, если $\lim_{x \to -\infty} f(x) = \infty $
}

\mcdfn{Ограниченная функция}
{
    Функция называется ограниченной при $x \to x_0$, если $\exists \delta > 0 \, \exists C > 0 \, \forall x \in \dot{U}_\delta(x_0): | f(x) | < C $
}

\mcdfn{Отделимая от нуля функция}
{
    Функция называется отделимой от нуля при $x \to x_0$, если 
    $\exists \delta > 0 \, \exists \veps_0 > 0 \, \forall x \in \dot{U}_\delta(x_0): | f(x) | > \veps_0 $ 
}


\nt{
    Связь функций при $ x \to x_0 $, где $ x $ - аргумент обоих функций, 
    $ x_0 $ - число, к которому стремится аргумент обоих функций:

\begin{tabular}{rl}

    & $\bullet$ $ \frac{1}{\text{б.б.}} = \text{б.м.} $ \\
    & $\bullet$ $ \frac{1}{\text{б.м.}} = \text{б.б.} $ \\
    & $\bullet$ $ \frac{1}{\text{ограниченная}} = \text{отделимая от нуля} $ \\
    & $\bullet$ $ \frac{1}{\text{отделимая от нуля}} = \text{ограниченная} $ \\
\end{tabular}
}

\section{Теорема о зажатой функции}

\mcthm{Теорема о зажатой функции}
{
\begin{minipage}[t]{\textwidth}
$$
\left. \begin{tabular}{l}
    $ f(x): \mathbb{R} \to \mathbb{R}, g(x): \mathbb{R} \to \mathbb{R}, h(x): \mathbb{R} \to \mathbb{R} $ \\
    $ \lim_{x \to x_0} f(x) = A $ \\
    $ \lim_{x \to x_0} h(x) = A $ \\
    $ \exists \delta > 0 \, \forall x \in \dot{U}_\delta(x_0): f(x) \le g(x) \le h(x) $
\end{tabular} \right\}
\begin{tabular}{l}
    $ \lim_{x \to x_0} g(x) = A $
\end{tabular}
$$
\end{minipage}
}

\section{Первый и второй замечательные пределы}

\mcdfn{Первый замечательный предел}
{
    \[ \lim_{x \to +\infty} \frac{\sin x}{x} = 1 \]
}

\mcdfn{Второй замечательный предел}
{
    \[ \lim_{x \to +\infty} \left(1 + \frac{1}{x}\right)^x = e \]
}

\section{Теорема о пределе сложной функции}

\mcthm{Теорема о пределе сложной функции}
{
$$
\left. \begin{tabular}{l}
    $ \lim_{x\to x_0} f(x) = y_0 $ \\
    $ \lim_{y\to y_0} g(y) = g(y_0) $
\end{tabular} \right\}
\begin{tabular}{l}
    $ \implies \lim_{x \to x_0} g(f(x)) = g(y_0) $
\end{tabular}
$$
\begin{mcproof}
\begin{equation*}
\begin{split}
& \text{Распишем, что дано, по определению:} \\
& \forall \veps > 0 \, \exists \delta_1(\veps) \, \forall x \in \dot{U}_{\delta_1(\veps)}(x_0): | f(x) - y_0 | < \veps \, (1) \\
& \forall \lambda > 0 \, \exists \delta_2(\lambda) \, \forall y \in \dot{U}_{\delta_2(\lambda)}(y_0): | g(y) - g(y_0) | < \lambda \, (2) \\
& \text{Распишем, что хотим доказать:} \\
& \forall \eta > 0 \, \exists \delta_3 = \delta(\eta) \forall x \in \dot{U}_{\delta_3(\eta)}(x_0): | g(f(x)) - g(y_0) | < \eta \\
& \text{Положим } \delta_3(\eta) = \delta_1(\delta_2(\eta))\text{, тогда}: \\
& x \in \dot{U}_{\delta_3(\eta)}(x_0) \iff x \in \dot{U}_{\delta_1(\delta_2(\eta))}(x_0) \implies \text{ по (1) } | f(x) - y_0 | < \delta_2(\eta) \\
& | f(x) - y_0 | < \delta_2(\eta) \iff f(x) \in U_{\delta_2(\eta)}(y_0) \\
& \text{По (2) знаем, что если } f(x) \in \dot{U}_{\delta_2(\eta)}(y_0) \text{, то } | g(f(x)) - g(y_0) | < \eta \\
& \text{Если } f(x) = y_0 \text{, то } | g(f(x)) - g(y_0) | = 0 < \eta \\
& \text{Иначе, если } f(x) \ne y_0 \iff f(x) \in \dot{U}_{\delta_2(\eta)}(y_0) \text{, то } | g(f(x)) - g(y_0) | = 0 < \eta \\
& \text{Получили: } \forall \eta > 0 \, \exists \delta_3 = \delta_1(\delta_2(\eta)) \, \forall x \in \dot{U}_{\delta_3(\eta)}(x_0): | g(f(x)) - g(y_0) | < \eta \\
\end{split}
\end{equation*}
\end{mcproof}
}

%
% if lim != g(y_0), then lets take: f(x) === 0, g(y) = y == 0; f(x) -> 0 = y_0 when x -> 0; g(y) -> 0 = A when y -> y_0; g(f(x)) === 1 -> 1 = A when x -> x_0
%

