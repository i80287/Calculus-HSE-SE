\chapter{Логические операции}


\section{Высказывания, предикаты и кванторы}

\subsection{Определения}

\mcdfn{Высказывания и n-местные предикаты}
{
    Высказывание - это упрощённая модель повествования предложения,
    такая что каждое высказывание либо истинно, либо ложно, но не одновременно

    n-местные предикат (n-арный предикат) - это выражение, которое превращается в
    высказывание, если в нём заменить $ x_1, x_2, ..., x_n $ на подходящие имена, где
    $ x_1, x_2, ..., x_n $ - переменные в предикате
}

\mcdfn{Логические операции}
{
\begin{tabular}{rl}
    Отрицаниe:       & $\bullet$ $ \lnot A $ (также обозначают $ \overline{A} $) означает "не $A$" \\
    Логическое и:    & $\bullet$ $ A \wedge B $ означает "верно $A$ и верно $B$" \\
    Логическое или:  & $\bullet$ $ A \vee B $ означает "верно $A$, или верно $B$, или верны $A$ и $B$ вместе" \\
    Исключающее или: & $\bullet$ $ A \oplus B $ означает "верно ровно одно из высказываний $A$, $B$" \\
    Импликация:      & $\bullet$ $ A \implies B $ означает "если верно $A$, то верно $B$" \\
    Эквивалентность: & $\bullet$ $ A \iff B $ означает "$A$ верно тогда и только тогда, когда верно $B$" \\
\end{tabular}
}

\nt{
    Пусть $ A \implies B $

    Если $ A $ верно, то $ B $ тоже верно, но если $ A $ ложно, то $ B $ может быть и истинным, и ложным

    Пусть $ A \iff B $

    Если $ A $ ложно, то ложно $ B $. Если $ B $ верно, то верно $ A $
}

\nt
{
    Логические операции можно выражать через другие логические операции, например,

    $ (A \implies B) \iff (\lnot A \vee B) $
}

\mcdfn{Кванторы}
{
Квантор всеобщности обозначается как $ \forall $ и означает "для любого"

Квантор существования обозначается как $ \exists $ и означает "существует"

Квантор едиственности обозначается как $ ! $ и означает "едиственный, такой что ..."
}

\mcex{}
{
\begin{tabular}{rl}
    Всеобщность:    & $\bullet$ $ \forall x \in \Rset: \, \phi(x) $ означает \\
                    & "Для любого x из $\Rset $ выполняется предикат $ \phi(x) $" \\
    Существование:  & $\bullet$ $ \exists x \, (x \in \mathbb{Q} \implies \psi(x)) $ означает \\
                    & "Существует x, такой что если x из $\mathbb{Q} $, то выполняется предикат $ \psi(x) $" \\
    Единственность: & $\bullet$ $ \forall n \in \mathbb{N} \, \exists! k \in \mathbb{N} \cup \{ 0 \}: 2^k \le n < 2^{k + 1} $ означает \\
                    & "Для любого натурального числа существует и едиственно такое \\
                    & целое неотрицательное число $ k $, что $ 2^k \le n < 2^{k + 1} $" \\
\end{tabular}
}

\nt{
    На практике квантор едиственности часто используется вместе с квантором
    существования т.е. часто используют связку $ \exists! $, "существует и единственно"
}

\nt{
    Вместо "$\lnot \exists$" пишут "$\nexists$"
}

\subsection{Правило обращения кванторов}

\mcclm{Правило обращения кванторов}{}
{
    При обращении кванторов квантор существования
    меняется на квантор всеобщности, квантор всеобщности
    меняется на квантор существования, а утверждение под
    кванторами меняется на противоположное
}

\mcex{}
{
    Пусть дано высказывание:

    \[
    \forall n \in \mathbb{N} \,
    \exists m_1 \in \mathbb{Z} \,
    \exists m_2 > m_1 \,
    \forall q \in \mathbb{Q}: \,
        | m_1 | > n \wedge \lnot
        \psi(q \cdot m_1 \cdot m_2 - n)
    \]

    Тогда отрицание к этому высказыванию будет:
    \[
    \exists n \in \mathbb{N} \,
    \forall m_1 \in \mathbb{Z} \,
    \forall m_2 > m_1 \,
    \exists q \in \mathbb{Q}: \,
        | m_1 | \le n \vee
        \psi(q \cdot m_1 \cdot m_2 - n)
    \]
}

\section{Метод математической индукции}

\mcclm{Метод математической индукции}{}{
    Пусть есть предикат $ \phi(n) $, который выполняется или не выполняется при различных $ n \in \mathbb{N} $

    Тогда, если $ \exists k \in \mathbb{N}: \, \phi(k) $ и
    $ \forall n \ge k: (\phi(n) \implies \phi(n + 1)) $, то
    по методу математической индукции получаем $ \forall n \ge k: \phi(n) $

    Этапы доказательства:

\begin{tabular}{rl}
    База индукции:          & $\bullet$ Проверка истинности $ \phi(k) $ \\
    Предположение индукции: & $\bullet$ Пусть для некоторого $ n \in \mathbb{N} \wedge n \ge k $ верно $ \phi(n) $ \\
    Шаг индукции:           & $\bullet$ Докажем, что $ \phi(n + 1) $, используя предположение индукции \\
    Вывод:                  & $\bullet$ $ \forall n \ge k: \phi(n) $ \\
\end{tabular}
}

\section{Неравенство Бернулли}

\mcthm{Неравенство Бернулли}{
    Если $ n \in \mathbb{N} $ и $ x \ge -1 $, то $ (1 + x)^n \ge 1 + xn $
\mcprf{
\begin{split}
    & \text{Докажем неравенство при помощи метода математической индукции} \\
    & \text{1. База индукции:} \\
    & \text{Пусть } n = 1 \implies (1 + x)^n = 1 + x \ge 1 + x \\
    & \text{2. Предположение индукции:} \\
    & \text{Пусть для некоторого } n \ge 1 \text{ верно, что } (1 + x)^n \ge 1 + xn \\
    & \text{3. Шаг индукции: Рассмотрим неравенство, подставив в него } n + 1: \\
    & (1 + x)^{n+1} = (1 + x)^n \cdot (1 + x) \\
    & 1 + x \ge 0 \implies (1 + x)^n \cdot (1 + x) \ge (1 + xn) \cdot (1 + x) = 1 + xn + x + n \cdot x^2 \ge 1 + nx + x = 1 + n(x + 1) \\
    & \text{Следовательно, } (1 + x)^{n+1} \ge 1 + n(x + 1) \\
    & \text{4. Обозначим доказываемое как предикат } \phi(n) \text{, тогда получаем:} \\
    & \phi(1) \wedge \forall n \in \mathbb{N}: (\phi(n) \implies \phi(n + 1)) \\
    & \text{Тогда по принципу математической индукции } \forall n \in \mathbb{N}: \phi(n) \\
\end{split}
}
}

\section{Перестановки, размещения, сочетания}

\mcdfn{Перестановки, размещения и сочетания}
{
    Пусть дано множество из $n$ элементов

    $\bullet$ Если все элементы попарно различны (т.е. при решении задачи мы считаем, что два любых элемента множества различны),
    то количество попарно различных перестановок этого множества обозначается как $ P_n $ и равно $ n! $

    Пусть зафиксировано $ k \in \mathbb{N} \cup \{ 0 \} $, такое что $ k \le n $, тогда:

    $\bullet$ Количество количество способов, которыми мы можем выбрать
    $k$-элементное подмножество данного множества, считая, что элементы попарно различны, 
    обозначается как $A_n^k$ и равно $\frac{n!}{(n - k)!}$

    $\bullet$ Количество количество способов, которыми мы можем выбрать
    $k$-элементное подмножество данного множества, считая, что все элементы попарно равны, 
    обозначается как $C_n^k$ и равно $\frac{n!}{k! (n - k)!}$
}

\nt{
    Пусть есть есть конечная последовательность из $ n $ натуральных чисел от 1 до $n$ (кортеж из $n$ элементов от $1$ до $n$)

    Тогда количество различных перестановок элементов кортежа равно $ P_n = n! $

    Количество способов выбрать $k$ чисел из кортежа, считая их перестановки различными, равно $ A_n^k = \frac{n!}{(n - k)!} $

    Количество способов выбрать $k$ чисел из кортежа, считая, что все перестановки одного набора - это один способ, равно $ C_n^k = \frac{n!}{k! (n - k)!} $

    Пусть $ \sigma = (1, 2, 3, 4) $ - данный кортеж, тогда есть $P_4 = 24$ различных перестановок $ \sigma $:
\begin{equation*}
\begin{split}
    & (1, 2, 3, 4), (1, 2, 4, 3), (1, 3, 2, 4), (1, 3, 4, 2), (1, 4, 2, 3), (1, 4, 3, 2) \\
    & (2, 1, 2, 4), (2, 1, 4, 2), (2, 3, 1, 4), (2, 3, 4, 1), (2, 4, 1, 3), (2, 4, 3, 1) \\
    & (3, 1, 2, 4), (3, 1, 4, 2), (3, 2, 1, 4), (3, 2, 4, 1), (3, 4, 1, 2), (3, 4, 2, 1) \\
    & (4, 1, 2, 3), (4, 1, 3, 2), (4, 2, 1, 3), (4, 2, 3, 1), (4, 3, 1, 2), (4, 3, 2, 1) \\
\end{split}    
\end{equation*}

    Для $k = 2$ есть $A_4^2 = 12 $ способ выбрать кортеж из 2 элементов:
\begin{equation*}
\begin{split}
    & (1, 2), (1, 3), (1, 4), (2, 1), (2, 3), (2, 4) \\
    & (3, 1), (3, 2), (3, 4), (4, 1), (4, 2), (4, 3) \\
\end{split}    
\end{equation*}

    Для $k = 2$ есть $C_4^2 = 6 $ способ выбрать подмножество из 2 элементов (порядок элементов не важен):
\begin{equation*}
\begin{split}
    & (1, 2), (1, 3), (1, 4), (2, 3), (2, 4), (3, 4) \\
\end{split}    
\end{equation*}
}

\section{Бином Ньютона}

\mcthm{Бином Ньютона}
{
    $ (a + b)^n = \sum_{k = 0}^{n} \Cmb{n}{k} a^{k} b ^ {n-k}
    (\text{формально, перед равенством необходимо написать }
        \forall a, b \in \Rset \forall n \in \mathbb{N}) $
\begin{mcproof}
\begin{equation*}
\begin{split}
& \text{Докажем это утверждение при помощи метода математической индукции} \\
& \text{1. База индукции: } n = 1 \implies (a + b)^n = a + b = \sum_{k=0}^{1} \Cmb{n}{k} a^k b^{n-k} \\
& \text{2. Предположение индукции: пусть для некоторого } n \ge 1: (a + b)^{n}= \sum_{k = 0}^{n} \Cmb{n}{k} a^{k} b ^ {n-k} \\
& \text{3. Рассмотрим равенство и докажем, что оно верно при подстановке } n + 1: \\
& (a + b)^{n + 1}
    = (a + b) (a + b)^n
    = (a + b) \sum_{k = 0}^{n} \Cmb{n}{k} a^{k} b ^ {n-k} = \\
&   = a \sum_{k = 0}^{n} \Cmb{n}{k} a^{k} b ^ {n - k}
        + b \sum_{k = 0}^{n} \Cmb{n}{k} a^{k} b ^ {n - k}
    = \sum_{k = 0}^{n} \Cmb{n}{k} a^{k + 1} b ^ {n - k}
        + \sum_{k = 0}^{n} \Cmb{n}{k} a^{k} b ^ {n + 1 - k} \\
&   = \sum_{k = 1}^{n + 1} \Cmb{n}{k - 1} a^{k} b ^ {n - (k - 1)}
        + \sum_{k = 0}^{n} \Cmb{n}{k} a^{k} b ^ {n + 1 - k}
    = \Cmb{n}{n} a^{n + 1} b^0 + \sum_{k = 1}^{n} \Cmb{n}{k - 1} a^{k} b ^ {n + 1 - k}
        + \Cmb{n}{0} a^{0} b^{n + 1} \sum_{k = 1}^{n} \Cmb{n}{k} a^{k} b ^ {n + 1 - k} = \\
&   = a^{n + 1} + b^{n + 1}
        + \sum_{k = 1}^{n} (\Cmb{n}{k - 1} + \Cmb{n}{k}) a^{k} b ^ {n + 1 - k}
    = \Cmb{n + 1}{n + 1} a^{n + 1} + \Cmb{n + 1}{0} b^{n + 1}
        + \sum_{k = 1}^{n} \Cmb{n + 1}{k} a^{k} b ^ {n + 1 - k} = \\
&   = \sum_{k = 0}^{n + 1} \Cmb{n + 1}{k} a^{k} b ^ {n + 1 - k} \\
& \text{4. Получили: } \\
& \text{Равенство верно при $n = 1$,
    а из верности равенства для $ n $ следует верность равенства для $ n + 1 $} \\
& \text{(при $ n \ge 1 $), тогда по методу математической индукции получим, что равенство верно $ \forall n \in \mathbb{N} $} \\
\end{split}
\end{equation*}
\end{mcproof}
}
