\chapter{Используемые обозначения}

\nt{
    $ \Nset $ - множество натуральных чисел. В данном файле полагаем, что $ 0 \notin \Nset $

    $ \Zset $ - множество целых чисел

    $ \Qset $ - множество рациональных чисел

    $ \Rset $ - множество вещественных чисел

    $ \Rset_{>0} $ - множество положительных вещественных чисел

    $ \ExtRset = \Rset \cup \{ +\infty; -\infty \} $ - дополненная прямая (extended real number line)

    $ \{ n, n + 1, ..., m \} $ - множество вещественных чисел от $n$ до $m$ "с шагом 1" включительно

    Формально, это множество равно $ \{ x \in \Rset | x \ge n \wedge x \le m \wedge x - n \in \Zset \}  $ 

    $ \circled{$\mathbb{W}$} $ - противоречие (используется при доказательстве методом от противного)

    (если Вам знаком символ $\bot$, то в данном файле полагаем эти обозначения эквивалентными)

    Через "$:=$" будем обозначать "положим по определению равным", например,
    $\forall \veps > 0:  \delta(\veps) := 2 \veps $ означает "для любого $\veps$, большего нуля, положим $\delta(\veps)$ равным $2 \veps$"

    Аббревиатурой БОО будем обозначать "без ограничения общности" 
    (возможно, Вам также знакомо выражение "не умаляя общности", в данном файле полагаем эти обозначения эквивалентными)

    $ C[a; b] $ - множество функций, непрерывных на отрезке $[a; b]$

    $ R[a; b] $ - множество функций, интегрируемых на отрезке $[a; b]$
}
